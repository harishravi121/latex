

$$time=(N*mtotal/mrelevant*x/n)*t/s$$

Intelligence quotient is proportional to 1/time.

\end{abstract}

\section{Intro}

speedofbrain=120/0.1;
speedofmuscle=40/0.2;
threshold=0.7;
we can remember $10^6$ pictures , $10^6$ audio sentances, $10^6$ tastes smells, touches
mem=1000000e6;  A million MB or a tera byte
Alu=10e8;
relavantmemory=100e6;
complexity=100;

intelligence=Percentageofcorrectdecisions/time

bloodlevel=50; ug/ml

speedofbrain=s0-deltaS*bloodlevel
threshold=t0+deltat*bloodlevel

It also affects the number of push ups a guy can do, the speed at which  a person can run etc. Thus a high dosage of depakote would increase the thresholds but reduce speed very much leading to lesser IQ. A very low dosage would lead to higher faulty decisions and a manic situation and less IQ. A sweet spot is dependent on the patient weight and past exposure to life and present exposure. The physical stamina would also be lower because the muscle needs higher rep rate and stable feedback. High dosage means lower rep rate or lesser weightbearing ability. Low dosage means incoherent behaviour between adjescent neurons. \textbf{Thus, medication reduces stamina and intelligence}.

stamina=reprate*coherenceinfeedbackbetweenadjescentmuscles

 

\begin{figure}
\centering
\includegraphics[width=0.7\columnwidth]{brain.png}
\caption{\label{fig:Lorenz}Intellignence and depakote blood level (theory)}
\end{figure}

import matplotlib.pyplot as plt\\
import numpy as np\\
speedofbrain=120/0.1;\\
speedtolocomotion=40/0.4;\\
threshold1=0.7;\\
deltathreshold=0.1;\\
bl=np.arange(5,150,5);\\
deltaS=200;\\
bloodlevel=bl/50;\\
threshold=threshold1+deltathreshold*bloodlevel;\\
speed=speedofbrain-deltaS*bloodlevel;\\
Ngenerations=4096;\\

imagine a binary tree firing voltagee 0.7 plus noise, next neuron firing probability is relativethreshold\\
n operations\\

prob=np.power((threshold1/threshold),12)\\
intelligence=prob*speed\\
prob2=1-np.power((threshold/threshold1),12)\\
intelligence=prob*speed\\

plt.plot(bl,intelligence)\\
plt.ylabel('Intelligence')\\
plt.xlabel('Blood level (uG/ml)')\\
plt.show()\\

If upon dosing with 50 uG/ml if the intelligence level is half, then if the medication is suddenly halved,
you would expect the intelligence to go up. On the contrary, because the brain is tuned to higher thresholds, there will be epilliptic thinking with own loops where the patient makes up his own stories about the situation and comes up to crazy conclusions. This means the IQ sporadically drops down especially while eating etc. Even if the patient is exposed to caffine which reduces the thresholds and increases speed, the same would happen. One may be able to reach higher levels of learning but hitting the right conclusions would be harder. One can't perform better if one goes on caffine. Thus tapering off medicene is a slow process where one tapers by steps of say 25 percent and then learns and lives in a normal fasion able to maintain basic hygine over a period of a week, eat enough to get consistent energy and go perform well at the job real time, maintain basic courtesies of a relation and not tread on the other. This relearning of basics and real life stuff will take about a few months before going to the next stage and so on. Going near the final stages will be very difficult because small changes in concentration result in big changes in decision making patterns. Will the addition of another drug cause a change to help taper we are not sure. Every drug has its charecteristic half life and mechanism of action by changning thresholds, speeds and neural spikes. Oleanz decreases IQ more drastically slowing down and increasing threshold more. It also affects the harmonal balances of dopamine etc which means the reward centres, pleasure centers light up less. Different schedule H drugs act on neurons or on the neuro harmonal systems. Harmonal systems act by activating by catalysis the synapses which means they act more exponentially as when compared to neural medicenes. Thus these neural medicenes calm the person for basic living and staying at home when a care giver is there. However, to make them able enough to work is difficult. One would have to try the same problem repeatedly and also remember to try again over the next day. To taper off when all situations are calm, one can do it when away from family to keep relations secure. Neuroplasticity gives hope provided one relearns by watching movies and travelling before joining a job. The blood level for depakote with half life is given by
$$Bloodlevel=Dosagemg/(Weightkg*0.068)*(1+thalfhrs/24*(1-exp(24/thalfhrs))$$

The base of the question is how does a man differ from another man when it comes to decision making and speed? It just comes from travel, exposure and education. If one is not behaving well according to circumstance they are subject to medication. However, with travel and enough exposure to experiences through discussions and audio visuals, one may switch to natural healing long term :)

To diagnose disorders based amount of speech (S) and eye contact (E) on a scale of 0 to 1, 1- the intersection gives anxiety if hez talking to someone he doesn't know or not properly to the stature, 1-union means hez talking and interacting too less means psychosis, 1-special bayes of eyecontact by talking gives bipolar and 1- special bayes of talking by eyecontact means obsessive compulsive disorder.

Bipolar $1-(P(E)+P(S)-P(E)P(S))/P(S))$
OCD $1-(P(E)+P(S)-P(E)P(S))/P(E))$
Anxiety $1-P(E)P(S)$ 
Toy fear or psychosis $1-(P(E)+P(S)-P(E)P(S))$ bingo

Future work based on me reading cases or books would be to recommend medicee dosage based on these umbers

\section{Speaking double sentences and imagination}

When the brain has different areas, if the speech originates from ideas in one area and a delayed area, the person speaks out two sentences with what one means to communicate and something more sutle and personal and this allows communication of more personal and other things which involve some shame. One needs to read a lot and practice the art of talking to someone else while readi

\end{document}
